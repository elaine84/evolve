In this work, we provided simple evolutionary mechanisms for evolving sparse
linear functions, under a large class of distributions. These evolutionary
mechanisms have the desirable properties that the representations used are
themselves sparse linear funcitons, and that they are attribute-efficient in the
sense that the number of generations required for evolution to succeed is
independent of the total number of attributes.

Strong negative results are known for distribution-independent evolvability of
boolean functions, \eg even the class of conjunctions is not
evolvable~\cite{Feldman:2011-LTF}.  However, it would interesting to study
whether under restricted classes of distributions, evolution is possible for
simple concept classes, using representations of low-complexity. Currently, even
under (biased) product distributions, no evolutionary mechanism is known for the
class of disjunctions, except via Feldman's general reduction from CSQ
algorithms. Even if the queries made by the CSQ algorithm are simple, Feldman's
reduction uses intermediate representations that randomly combine queries made
by the algorithm. 

A natural extension of our current results would be to fixed-degree sparse
polynomials. An interesting class of boolean functions is low-weight threshold
functions, which includes disjunctions and conjunctions.  The class of smooth
bounded distributions may be a natural starting place for studying evolvability
of simple concept classes. For example, is the class of low-weight threshold
functions evolvable under smooth distributions, or at least log-concave
distributions? 
